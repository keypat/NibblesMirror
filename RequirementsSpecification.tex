\documentclass[12pt]{article}

\usepackage{bm}
\usepackage{amsmath, mathtools}
\usepackage{amsfonts}
\usepackage{amssymb}
\usepackage{graphicx}
\usepackage{colortbl}
\usepackage{xr}
\usepackage{hyperref}
\usepackage{longtable}
\usepackage{xfrac}
\usepackage{tabularx}
\usepackage{float}
\usepackage{siunitx}
\usepackage{booktabs}
\usepackage{caption}

%\usepackage{refcheck}

\hypersetup{
    bookmarks=true,         % show bookmarks bar?
      colorlinks=true,       % false: boxed links; true: colored links
    linkcolor=red,          % color of internal links (change box color with linkbordercolor)
    citecolor=green,        % color of links to bibliography
    filecolor=magenta,      % color of file links
    urlcolor=cyan           % color of external links
}
\newcommand{\NN}[1]{{\color{red}#1}}
\newcommand{\WSS}[1]{{\color{blue}#1}}

\newcommand{\colZwidth}{1.0\textwidth}
\newcommand{\blt}{- } %used for bullets in a list
\newcommand{\colAwidth}{0.13\textwidth}
\newcommand{\colBwidth}{0.82\textwidth}
\newcommand{\colCwidth}{0.1\textwidth}
\newcommand{\colDwidth}{0.05\textwidth}
\newcommand{\colEwidth}{0.8\textwidth}
\newcommand{\colFwidth}{0.17\textwidth}
\newcommand{\colGwidth}{0.5\textwidth}
\newcommand{\colHwidth}{0.28\textwidth}
\newcounter{defnum} %Definition Number
\newcommand{\dthedefnum}{GD\thedefnum}
\newcommand{\dref}[1]{GD\ref{#1}}
\newcounter{datadefnum} %Datadefinition Number
\newcommand{\ddthedatadefnum}{DD\thedatadefnum}
\newcommand{\ddref}[1]{DD\ref{#1}}
\newcounter{theorynum} %Theory Number
\newcommand{\tthetheorynum}{T\thetheorynum}
\newcommand{\tref}[1]{T\ref{#1}}
\newcounter{tablenum} %Table Number
\newcommand{\tbthetablenum}{T\thetablenum}
\newcommand{\tbref}[1]{TB\ref{#1}}
\newcounter{assumpnum} %Assumption Number
\newcommand{\atheassumpnum}{P\theassumpnum}
\newcommand{\aref}[1]{A\ref{#1}}
\newcounter{goalnum} %Goal Number
\newcommand{\gthegoalnum}{P\thegoalnum}
\newcommand{\gsref}[1]{GS\ref{#1}}
\newcounter{instnum} %Instance Number
\newcommand{\itheinstnum}{IM\theinstnum}
\newcommand{\iref}[1]{IM\ref{#1}}
\newcounter{reqnum} %Requirement Number
\newcommand{\rthereqnum}{P\thereqnum}
\newcommand{\rref}[1]{R\ref{#1}}
\newcounter{lcnum} %Likely change number
\newcommand{\lthelcnum}{LC\thelcnum}
\newcommand{\lcref}[1]{LC\ref{#1}}

\newcommand{\tclad}{T_\text{CL}}
\newcommand{\degree}{\ensuremath{^\circ}}
\newcommand{\progname}{SWHS}

%\oddsidemargin 0mm
%\evensidemargin 0mm
%\textwidth 160mm
%\textheight 200mm
\usepackage{fullpage}

\begin{document}

\title{Software Requirements Specification
SNAKE
} 
\author{Alex Guerrero, Keyur Patel and Shafeeq Rabbani}
\date{\today}
	
\maketitle

\tableofcontents

\section{Reference Material}

This section records information for easy reference.

\subsection{Table of Units}

Not applicable.
~\newline
\renewcommand{\arraystretch}{1.2}
\subsection{Table of Symbols}
Not applicable
%~\newline
\subsection{Abbreviations and Acronyms}
\renewcommand{\arraystretch}{1.2}

\section{Introduction}

The scope of the requirements is limited to...

\subsection{Purpose of Document}

		The purpose of Snake is to help users have fun. Snake is designed to engage users in an enjoyable activity which encourages them to use strategic planning and time management to achieve higher and higher scores.
		
\subsection{Scope of Requirements} 

The scope of the requirements is limited to...

\subsection{Organization of Document}

The template for this document follows the SRS for Volere template.

%\subsection{Intended Audience}

\section{General System Description}

This section provides general information about the system,
identifies the interfaces between the system and its environment, and describes the user characteristics and the system constraints.

%\subsection{System Context}

\subsection{User Characteristics}

The end user should be familiar with the basics of using a computer enough to install python and pygame using instructions outlined on their respective websites.

\subsection{System Constraints}

There are no system constraints.

\section{Specific System Description}

This section first presents the problem description, which gives a high-level
view of the problem to be solved.  This is followed by the solution characteristics
specification, which presents the assumptions, theories, definitions and finally
the instance models (ODEs) that model the solar water heating tank with PCM.

\subsection{Problem Description} \label{Sec_pd}

The traditional snake game is to be built on to include additional features which are more relatable to today’s gamers.

%\subsubsection{Background}

\subsubsection{Terminology and  Definitions}

This subsection provides a list of terms that are used in the subsequent
sections and their meaning, with the purpose of reducing ambiguity and making it
easier to correctly understand the requirements:

\begin{itemize}

\item Client: Administrative staff of SFWR ENG 3XA3 (Doctor Smith and TAs)
		 	 	 						
\item Constraint: A global requirement that affects decisions about the scope of the project. 
\item Stakeholder: A person, group or organization that has interest or concern in the project.

\item Player: A person who plays the video game.

\item Snake: The classic video game conceptualized in the late 1970’s, where a player manoeuvres a line which grows in length, with the line itself being a primary obstacle
				
\end{itemize}

\subsubsection{Physical System Description}


There are no physical portions to this project.



\subsubsection{Goal Statements}

\noindent Given the temperature of the coil, initial conditions for the temperature of 
the water and the PCM, and material properties, the goal statements are:

\begin{itemize}

\item[GS\refstepcounter{goalnum}\thegoalnum:] The game must serve as a source of entertainment for the user.

\item[GS\refstepcounter{goalnum}\thegoalnum:] The game must consist of additional features to what is traditionally included in the snake game.
	
\item[GS\refstepcounter{goalnum}\thegoalnum:] The game must have ample documentation and well commented code so that the program is maintainable and expandable in the future.

\end{itemize}

\subsection{Solution Characteristics Specification}


\subsubsection{Assumptions}


\section{Requirements}

This section provides the functional requirements, the business tasks that the
software is expected to complete, and the nonfunctional requirements, the
qualities that the software is expected to exhibit.

\subsection{Functional Requirements}

\noindent \begin{itemize}

\item[R\refstepcounter{reqnum}\thereqnum:] The user must be able to start the game by a single click.

\item[R\refstepcounter{reqnum}\thereqnum:] The game must be controlled by keyboard.

\item[R\refstepcounter{reqnum}\thereqnum:] R3: When the snake hits an obstacle , freeze the game indicating game over.

\item[R\refstepcounter{reqnum}\thereqnum:] When the snake collects ‘food’ its size grows.
\item[R\refstepcounter{reqnum}\thereqnum:]  As the user advances in the game, the snake becomes faster.
\item[R\refstepcounter{reqnum}\thereqnum:] When the user goes off border, the snake appears from the other side.
\item[R\refstepcounter{reqnum}\thereqnum:] As an additional feature, when the snake goes through ‘portals’ it comes out from the other side.
\item[R\refstepcounter{reqnum}\thereqnum:] The game calculates a score that is based on the amount of time spent in the game and any ‘food’ collected.

\item[R\refstepcounter{reqnum}\thereqnum:] The game outputs the score after the game is finished.

\item[R\refstepcounter{reqnum}\thereqnum:] The game must respond instantly to user inputs without any lag.

\end{itemize}

\subsection{Nonfunctional Requirements}

\noindent \begin{itemize}

\item[R\refstepcounter{reqnum}\thereqnum:] The project must satisfy the requirements of the course.


\item[R\refstepcounter{reqnum}\thereqnum:] The code of the game must function correctly.

\item[R\refstepcounter{reqnum}\thereqnum:] R3: The code of the game must be verifiable (by means of testing?) 

\item[R\refstepcounter{reqnum}\thereqnum:] There must be ample documentation for the game and it must be well commented in order to allow future expandability, maintainability and reusability.

\end{itemize}


\section{Likely Changes}    



\bibliographystyle {plain}
\bibliography {PCM_SRS}

\end{document}
