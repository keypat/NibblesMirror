\documentclass[12pt]{article}

\usepackage{bm}
\usepackage{amsmath, mathtools}
\usepackage{amsfonts}
\usepackage{amssymb}
\usepackage{graphicx}
\usepackage{colortbl}
\usepackage{xr}
\usepackage{hyperref}
\usepackage{longtable}
\usepackage{xfrac}
\usepackage{tabularx}
\usepackage{float}
\usepackage{siunitx}
\usepackage{booktabs}
\usepackage{caption}

%\usepackage{refcheck}

\hypersetup{
    bookmarks=true,         % show bookmarks bar?
      colorlinks=true,       % false: boxed links; true: colored links
    linkcolor=red,          % color of internal links (change box color with linkbordercolor)
    citecolor=green,        % color of links to bibliography
    filecolor=magenta,      % color of file links
    urlcolor=cyan           % color of external links
}
\newcommand{\NN}[1]{{\color{red}#1}}
\newcommand{\WSS}[1]{{\color{blue}#1}}

\newcommand{\colZwidth}{1.0\textwidth}
\newcommand{\blt}{- } %used for bullets in a list
\newcommand{\colAwidth}{0.13\textwidth}
\newcommand{\colBwidth}{0.82\textwidth}
\newcommand{\colCwidth}{0.1\textwidth}
\newcommand{\colDwidth}{0.05\textwidth}
\newcommand{\colEwidth}{0.8\textwidth}
\newcommand{\colFwidth}{0.17\textwidth}
\newcommand{\colGwidth}{0.5\textwidth}
\newcommand{\colHwidth}{0.28\textwidth}
\newcounter{defnum} %Definition Number
\newcommand{\dthedefnum}{GD\thedefnum}
\newcommand{\dref}[1]{GD\ref{#1}}
\newcounter{datadefnum} %Datadefinition Number
\newcommand{\ddthedatadefnum}{DD\thedatadefnum}
\newcommand{\ddref}[1]{DD\ref{#1}}
\newcounter{theorynum} %Theory Number
\newcommand{\tthetheorynum}{T\thetheorynum}
\newcommand{\tref}[1]{T\ref{#1}}
\newcounter{tablenum} %Table Number
\newcommand{\tbthetablenum}{T\thetablenum}
\newcommand{\tbref}[1]{TB\ref{#1}}
\newcounter{assumpnum} %Assumption Number
\newcommand{\atheassumpnum}{P\theassumpnum}
\newcommand{\aref}[1]{A\ref{#1}}
\newcounter{goalnum} %Goal Number
\newcommand{\gthegoalnum}{P\thegoalnum}
\newcommand{\gsref}[1]{GS\ref{#1}}
\newcounter{instnum} %Instance Number
\newcommand{\itheinstnum}{IM\theinstnum}
\newcommand{\iref}[1]{IM\ref{#1}}
\newcounter{reqnum} %Requirement Number
\newcommand{\rthereqnum}{P\thereqnum}
\newcommand{\rref}[1]{R\ref{#1}}
\newcounter{lcnum} %Likely change number
\newcommand{\lthelcnum}{LC\thelcnum}
\newcommand{\lcref}[1]{LC\ref{#1}}

\newcommand{\tclad}{T_\text{CL}}
\newcommand{\degree}{\ensuremath{^\circ}}
\newcommand{\progname}{SWHS}

%\oddsidemargin 0mm
%\evensidemargin 0mm
%\textwidth 160mm
%\textheight 200mm
\usepackage{fullpage}

\begin{document}

\title{Software Requirements Specification
SNAKE
} 
\author{Alex Guerrero, Keyur Patel and Shafeeq Rabbani}
\date{\today}
	
\maketitle

\tableofcontents
\newpage


\section{Project Drivers}

The scope of the requirements is limited to...

\subsection{The Purpose of the Project}

		The purpose of Snake is to help users have fun. Snake is designed to engage users in an enjoyable activity which encourages them to use strategic planning and time management to achieve higher and higher scores.
		
\subsection{The Stakeholders and Users} 

The client for this project is the administrative staff that teach SFWR ENG 3XA3. This includes Dr. Smith, and the teaching assistants for the course. They are the recipients of the course work and are responsible for the supervision and evaluation of the product. Other stakeholders include the team members working on the project, and classmates who would want to critique and enjoy our program. The users of this product will be (potentially) available as open source software to the online community . This group of users can vary from beginner programmers to professionals in the computing industry. Beginner programmers are the lowest level of users because this product is delivered in a programming language and uses libraries that does not allow it to be packaged into an executable. Other users include the core members of the project, because of the sense of pride they will feel towards the product.

\section{Project Constraints}

\subsection{Mandated Constraints}

There is a time constraint for this product which says that it must be completed by the first week of December 2015. Another constraint states that the product must not be implemented on a mobile device, the rationale being that the professor and TAs may not have access to the devices used. Because the team members designing the product are university students, there is no funding and the budget must be limited to a total of 0.00 CAD. The current system is implemented on any personal computer or laptop which has 32-bit Python 2.7, and PyGame installed. Not other hardware restrictions are known in order to implement the current system.

\subsection{Naming Conventions and Terminology}

\begin{itemize}

\item Client: Administrative staff of SFWR ENG 3XA3 (Doctor Smith and TAs)
		 	 	 						
\item Constraint: A global requirement that affects decisions about the scope of the project. 
\item Stakeholder: A person, group or organization that has interest or concern in the project.

\item Player: A person who plays the video game.

\item Snake: The classic video game conceptualized in the late 1970’s, where a player manoeuvres a line which grows in length, with the line itself being a primary obstacle
				
\end{itemize}

\subsection{Relevant Facts and Assumptions}

The existing application only has 91 lines of Python code, which provides the barebones of the ideal snake game. There is no documentation or modularization of the code.

\section{Functional Requirements}
\subsection{The Scope of the Work}
The scope of the work is limited to a simple implementation of the game Snake
\subsection{Business Data Model and Data Dictionary}
Not applicable for this project
\subsection{The Scope of the Product}
Refer to The Scope of the Work
\subsection{Functional Requirements}
\begin{itemize}
\item R1: The user must be able to start the game by a single click.
\item R2: The snake must be controlled by keyboard.
\item R3: When the snake hits an obstacle , freeze the game indicating game over.
\item R4: When the snake collects ‘food’ its length grows by 1 unit.
\item R5: As the user advances in the game, the snake becomes faster.
\item R6: When the user hits the border, the snake dies.
\item R7: The game calculates a score that is based on the length of the snake.
\item R8: The game outputs the score after the game is finished.
\end{itemize}

\section{Non-functional Requirements} 
\subsection{Look and Feel Requirements}
\subsubsection*{Appearance Requirements}
\begin{itemize}
\item 1: the product must be able to display a main menu with buttons and lists
\item 2: the product must be able to display an error window if anything occurs
\item 3: the product should have a soundtrack playing during the main menu and gameplay
\end{itemize}

\subsubsection*{Style Requirements}
\begin{itemize}
\item 1: The product shall appear minimalistic in terms of graphics and visuals.
\end{itemize}

\subsection{Usability and Humanity Requirements}
\subsubsection*{Ease of Use Requirements}
\begin{itemize}
\item 1: the product must be easy to use by anyone from children to seniors (after installation of required software).
\item 2:  The casual user is expected to remember at most 7 keys an their functions
\end{itemize}

\subsubsection*{Personalization and Internationalization Requirements}
Not Applicable for this project

\subsubsection*{Learning Requirements}
\begin{itemize}
\item 1: The product shall be easy for a child to learn once the software requirements are met
\end{itemize}

\subsubsection*{Understandability and Politeness Requirements}
\begin{itemize}
\item 1: the product shall use common directional movement keys
\item 2: the product shall hide its implementation and construction from the user
\end{itemize}

\subsubsection*{Accessibility Requirements}
Not Applicable for this project

\subsection{Performance Requirements}

\subsubsection*{Speed and Latency Requirements}
\begin{itemize}
\item 1: The response to any keys shall be fast enough to avoid delayed actions (lag)
\end{itemize}

\subsubsection*{Safety Critical Requirements}
Not applicable for this project
\subsubsection*{Precision or Accuracy Requirements}
Not applicable for this project

\subsubsection*{Reliability and Availability Requirements}
\begin{itemize}
\item 1: The product shall be usable 24h of the day, as it does not require network connectivity
\end{itemize}

\subsubsection*{Robustness or Fault-Tolerance Requirements}
Not applicable for this project
\subsubsection*{Capacity Requirements}
Not applicable for this project
\subsubsection*{Scalability and Extensibility Requirements}
Not applicable for this project
\subsubsection*{Longevity Requirements}
Not applicable for this project

\subsection{Operational and Envrionmental Requirements}

\subsubsection*{Expected Physical Environment}
\begin{itemize}
\item 1: The product shall be usable in any environment where the computer it is on can be used
\end{itemize}

\subsubsection*{Requirements for Interfacing with Adjacent Systems}
Not Applicable to this project

\subsubsection*{Productization Requirements}
\begin{itemize}
\item 1: The product shall be distributed as a ZIP file
\item 2: The product shall be able to be installed by a user with the aid of instructions
\end{itemize}

\subsubsection*{Release Requirements}
Not Applicable for this project

\subsection{Maintainability and Support Requirements}

\subsubsection*{Maintenance Requirements}
\begin{itemize}
\item 1: The product shall be able to be maintained by developers who are not the original developers
\item 2: The product shall provide documentation for the code for other developers
\end{itemize}

\subsubsection*{Supportability Requirements}
Not Applicable for this project

\subsubsection*{Adaptability Requirements}
\begin{itemize}
\item 1: The product is expected to run on any OS which supports python 2.7
\end{itemize}

\subsection{Security Requirements}

\subsubsection*{Access Requirements}
Not Applicable for this project
\subsubsection*{Integrity Requirements}
Not Applicable for this project
\subsubsection*{Privacy Requirements}
Not Applicable for this project
\subsubsection*{Audit Requirements}
Not Applicable for this project
\subsubsection*{Immunity Requirements}
Not Applicable for this project

\subsection{Cultural Requirements}

\subsubsection*{Cultural Requirements}
\begin{itemize}
\item 1: The product shall not be offensive to religious or ethnic groups
\end{itemize}

\subsubsection*{Compliance Requirements}
Not Applicable for this project
\subsubsection*{Standards Requirements}
Not Applicable for this Project

\section{Project Issues}

\subsection{Open Issues}

\subsection{Off-the-Shelf Solutions}

\subsection{New Problems}

\subsection{Tasks}

\subsection{Migration of the New Product}

\subsection{Risks}

\subsection{Costs}

\subsection{User Documentation and Training}

\subsection{Waiting Room}
These are ideas which are going to be considered in the future but are not being currently developed on at the moment.

When the user collects a certain power up, the snake gets the intangibility power up allowing it to pass through all obstacles including its own body. This powerup will make it more convenient for the user to collect food and allow the user to do so more quickly.

The Sacrifice feature would allow the user to continue playing the game even though the snake is about to take the hit. It will do this by allowing the user to cut the length of the snake at the expense of some points.

The Freeze feature would slow the snake. This is a feature that is exceptionally handy when the snake is move at really fast at must later stages of the game.
\subsection{Ideas for Solutions}


%FOR REFERENCE
%\item[GS\refstepcounter{goalnum}\thegoalnum:] The game must serve as a source of entertainment for the user.




\bibliographystyle {plain}
\bibliography {PCM_SRS}

\end{document}
