\documentclass[12pt]{article}

\usepackage{bm}
\usepackage{amsmath}
\usepackage{amsfonts}
\usepackage{amssymb}
\usepackage{graphicx}
\usepackage{colortbl}
\usepackage{xr}
\usepackage{hyperref}
\usepackage{longtable}
\usepackage{xfrac}
\usepackage{tabularx}
\usepackage{float}
\usepackage{siunitx}
\usepackage{booktabs}

%\usepackage{refcheck}

\hypersetup{
    bookmarks=true,         % show bookmarks bar?
      colorlinks=true,       % false: boxed links; true: colored links
    linkcolor=red,          % color of internal links (change box color with linkbordercolor)
    citecolor=green,        % color of links to bibliography
    filecolor=magenta,      % color of file links
    urlcolor=cyan           % color of external links
}
%\newcommand{\wss}[1]{\authornote{magenta}{SS}{#1}}
%\newcommand{\hed}[1]{\authornote{blue}{HM}{#1}}
%\newcommand{\tz}[1]{\authornote{blue}{TZ}{#1}} 
%\newcommand{\pl}[1]{\authornote{blue}{PL}{#1}} 

\newcommand{\colZwidth}{1.0\textwidth}
\newcommand{\blt}{- } %used for bullets in a list
\newcommand{\colAwidth}{0.13\textwidth}
\newcommand{\colBwidth}{0.82\textwidth}
\newcommand{\colCwidth}{0.1\textwidth}
\newcommand{\colDwidth}{0.05\textwidth}
\newcommand{\colEwidth}{0.8\textwidth}
\newcommand{\colFwidth}{0.17\textwidth}
\newcommand{\colGwidth}{0.5\textwidth}
\newcommand{\colHwidth}{0.28\textwidth}
\newcounter{defnum} %Definition Number
\newcommand{\dthedefnum}{GD\thedefnum}
\newcommand{\dref}[1]{GD\ref{#1}}
\newcounter{datadefnum} %Datadefinition Number
\newcommand{\ddthedatadefnum}{DD\thedatadefnum}
\newcommand{\ddref}[1]{DD\ref{#1}}
\newcounter{theorynum} %Theory Number
\newcommand{\tthetheorynum}{T\thetheorynum}
\newcommand{\tref}[1]{T\ref{#1}}
\newcounter{tablenum} %Table Number
\newcommand{\tbthetablenum}{T\thetablenum}
\newcommand{\tbref}[1]{TB\ref{#1}}
\newcounter{assumpnum} %Assumption Number
\newcommand{\atheassumpnum}{P\theassumpnum}
\newcommand{\aref}[1]{A\ref{#1}}
\newcounter{goalnum} %Goal Number
\newcommand{\gthegoalnum}{P\thegoalnum}
\newcommand{\gsref}[1]{GS\ref{#1}}
\newcounter{instnum} %Instance Number
\newcommand{\itheinstnum}{IM\theinstnum}
\newcommand{\iref}[1]{IM\ref{#1}}
\newcounter{reqnum} %Requirement Number
\newcommand{\rthereqnum}{P\thereqnum}
\newcommand{\rref}[1]{R\ref{#1}}
\newcounter{lcnum} %Likely change number
\newcommand{\lthelcnum}{LC\thelcnum}
\newcommand{\lcref}[1]{LC\ref{#1}}

\newcommand{\tclad}{T_\text{CL}}
\newcommand{\degree}{\ensuremath{^\circ}}
\newcommand{\progname}{SWHS}

\usepackage{fullpage}

\begin{document}

\title{Test Plan} 
\author{Alex Guerrero, Keyur Patel and Shafeeq Rabbani}
\date{\today}

\maketitle

\tableofcontents
\newpage
%%%%%%%%%%%%%%%%%%%%%%%%
%
%	1.) General Information 
%
%%%%%%%%%%%%%%%%%%%%%%%%

\section{Introduction}
The test plan is designed to identify the types of tests to perform and helps explain how tests will be performed.

\subsection{Test Items}
The different items to be tested includes:

\begin{itemize}
\item A: The functions and methods of each class of the Model (backend)
\item B: The game board against the functional requirements of the product
\item C: The graphical interface that implements the Model 
\end{itemize}

\section{Software Risk Issues}

\section{Features to be Tested}

\section{Features not to be Tested}


\section{Testing Types}
Testing can be broken up into different types, which each have their own role in the testing the product. These test types should be utilized to comprehensively evaluate the quality of the product.
\subsection{Structural Testing}
Structural testing  is also known as white box testing. Structural tests are derived from the program's internal structure. It focuses on the nonfunctional requirements of the product. This type of testing shows errors that occur during the implementation by focusing on abnormal and extreme cases the product could encounter.
\subsection{Functional Testing}
Functional testing is also known as black box testing. Functional tests are derived from the functional requirements of the program. It focuses less on how the program works and more on the output of the system. These tests are focused on test cases where the product receives expected information.
\subsection{Static vs. Dynamic Testing}
Static testing simulate the dynamic environment and does not focus on code exectution. This testing involves code walkthroughs and requirements walkthroughs. Static testing is used prevalently in the design stage. In contrast, dynamic testing needs code to be executed. \newline\newline
Dynamic testing involves test cases to be run and checked against expected outcomes. A technique to save time during dynamic testing is to choose representative test cases. 
\subsection{Manual vs. Automatic Testing}
Manual testing is done by people. It involves code walkthroughs and inspection. \newline\newline
Automatic testing can usually be conducted by computers. The tools used to assist with automatic are unit testing tools for the respective programming language. Automatic testing relies on people for testing more qualitative aspects like GUI. 


\section{Approach}

\subsection{Testing functional requirements}
\textbf{THE REQUIREMENT}
\subsubsection*{Test Type}
\subsubsection*{Test Factors Involved}
\subsubsection*{Initial State}
\subsubsection*{Inputs}
\subsubsection*{Outputs}
\subsubsection*{Schedule}
\subsubsection*{Methodology}
\subsubsection*{Test For}



\subsection{Test Cases for Snake.py}

\begin{center}
	\begin{longtable}{ | r | p{4cm} | p{4cm} | p{4cm} |}
	\caption{Snake.py} \\ \hline \label{TblInputVar} 
	Method & Input & Expected Outcome\\ \hline
	constructor & none &  first 20 points of the snake are generated\\ \hline
	changeDir & dir=-1 & if current direction is 2 or -2, it will be updated to -1\\ \hline
	\end{longtable}
\end{center}




\subsection{Test Cases for Map.py}
\subsection{Test Cases for Food.py}
\subsection{Testing for GUI}




\bibliographystyle {plain}
\bibliography {PCM_SRS}

\end{document}
